\documentclass[a4paper]{book}
\usepackage{fullpage}

\usepackage[utf8]{inputenc}
\usepackage[T1]{fontenc}
\usepackage[francais]{babel}

\usepackage{latexsym}
\usepackage{fancyhdr}
\usepackage{makeidx}
\usepackage{graphics}
\usepackage{graphicx}
\usepackage{longtable}
\usepackage{moreverb}
\usepackage{listings}

\newcommand{\altarica}{{\sc AltaRica}}

\begin{document}

\title{Master 1, Conceptions Formelles\\
Projet du module \altarica\\
Synthèse (assistée) d'un contrôleur du niveau d'une cuve}

\date{}

\author{Jolliet Louis \and Sicardon Louis \and Vigneau Paul}

\maketitle

\chapter{Le sujet}
\section{Cahier des charges}

Le système que l'on souhaite concevoir est composé~:
\begin{itemize}
\item d'un réservoir contenant {\bf toujours} suffisamment d'eau pour alimenter l'exploitation,
\item d'une cuve,
\item de deux canalisations parfaites amont reliant le réservoir à la cuve, et permettant d'amener l'eau à la cuve,
\item d'une canalisation parfaite aval permettant de vider l'eau de la cuve,
\item chaque canalisation est équipée d'une vanne commandable, afin de réguler l'alimentation et la vidange de la cuve,
\item d'un contrôleur.
\end{itemize}

\subsection{Détails techniques}

\subsubsection{La vanne}
Les vannes sont toutes de même type, elles possèdent trois niveaux de débits correspondant à trois diamètres d'ouverture~: 0 correspond à la vanne fermée, 1 au diamètre intermédiaire et 2 à la vanne complètement ouverte. Les vannes sont commandables par les deux instructions {\tt inc} et {\tt dec} qui respectivement augmente et diminue l'ouverture. Malheureusement, la vanne est sujet à défaillance sur sollicitation, auquel cas le système de commande devient inopérant, la vanne est désormais pour toujours avec la même ouverture.

\subsubsection{La Cuve}
Elle est munie de $nbSensors$ capteurs (au moins quatre) situés à $nbSensors$ hauteurs qui permettent de délimiter $nbSensors+1$ zones. La zone 0 est comprise entre le niveau 0 et le niveau du capteur le plus bas; la zone 1 est comprise entre ce premier capteur et le second, et ainsi de suite.

Elle possède en amont un orifice pour la remplir limité à un débit de 4, et en aval un orifice pour la vider limité à un débit de 2.  

\subsubsection{Le contrôleur}
Il commande les vannes avec les objectifs suivants ordonnés par importance~:
\begin{enumerate}
\item Le système ne doit pas se bloquer, et le niveau de la cuve ne doit jamais atteindre les zones 0 ou $nbSensors$.
\item Le débit de la vanne aval doit être le plus important possible.
\end{enumerate}

On fera également l'hypothèse que les commandes ne prennent pas de temps, et qu'entre deux pannes et/ou cycle {\em temporel}, le contrôleur à toujours le temps de donner au moins un ordre. Réciproquement, on fera l'hypothèse que le système à toujours le temps de réagir entre deux commandes.

\subsubsection{Les débits}
Les règles suivantes résument l'évolution du niveau de l'eau dans la cuve~:
\begin{itemize}
\item Si $(amont > aval)$ alors au temps suivant, le niveau aura augmenté d'une unité.
\item Si $(amont < aval)$ alors au temps suivant, le niveau aura baissé d'une unité.
\item Si $(amont = aval = 0)$ alors au temps suivant, le niveau n'aura pas changé.
\item Si $(amont = aval > 0)$ alors au temps suivant, le niveau pourra~:
  \begin{itemize}
  \item avoir augmenté d'une unité,
  \item avoir baissé d'une unité,
  \item être resté le même.
  \end{itemize}
\end{itemize}

\section{L'étude}

\subsection{Rappel méthodologique}
Comme indiqué en cours, le calcul par point fixe du contrôleur est exact, mais l'opération de projection effectuée ensuite peut perdre de l'information et générer un contrôleur qui n'est pas satisfaisant. Plus précisemment, le contrôleur \altarica\ généré~:
\begin{itemize}
\item ne garanti pas la non accessibilité des \emph{Situations Redoutées}.
\item ne garanti pas l'absence de \emph{nouvelles situations de blocages}.
\end{itemize}

Dans le cas ou il existe toujours \emph{des situations de blocages ou redoutées}, vous pouvez au choix~:
\begin{enumerate}
\item Corriger manuellement le contrôleur calculé (sans doute très difficile).
\item Itérer le processus du calcul du contrôleur jusqu'à stabilisation du résultat obtenu. 
  \begin{itemize}
  \item Si le contrôleur obtenu est sans blocage et sans situation redoutée, il est alors correct.
  \item Si le contrôleur obtenu contient toujours des blocages ou des situations redoutées, c'est que le contrôleur initial n'est pas assez performant, mais rien ne garanti que l'on soit capable de fournir ce premier contrôleur suffisemment performant.
  \end{itemize}
\end{enumerate}

{\bf Remarque} : Pour vos calculs, vous pouvez utiliser au choix les commandes~:
\begin{itemize}
\item {\tt altarica-studio xxx.alt xxx.spe}
\item {\tt arc -b xxx.alt xxx.spe}
\item {\tt make} pour utiliser le fichier GNUmakefile fourni.
\end{itemize}

\subsection{Le travail a réaliser}

Avant de calculer les contrôleurs, vous devez répondre aux questions suivantes.
\begin{enumerate}
\item Expliquez le rôle de la constante $nbFailures$ et de la contrainte, présente dans le composant {\tt System}, $nbFailures >= (V[0].fail + V[1].fail + V[2].fail)$.
\item Expliquez le rôle du composant {\tt ValveVirtual} et de son utilisation dans le composant {\tt CtrlVV}, afin de remplacer le composant {\tt Ctrl} utilisé initialement.
\end{enumerate}

L'étude consiste à étudier le système suivant deux paramètres~:
\begin{enumerate}
\item $nbFailures$~: une constante qui est une borne pour le nombre de vannes pouvant tomber en panne.
\item Le contrôleur initial qui peut être soit {\tt Ctrl}, soit {\tt CtrlVV}.
\end{enumerate}

Pour chacun des huit systèmes étudiés, vous devez décrire votre méthodologie pour calculer les différents contrôleurs et répondre aux questions suivantes~:

\begin{enumerate}
\item Est-il possible de contrôler en évitant les blocages et les situations critiques ?
\item Si oui, donnez quelques caractéristiques de ce contrôleur, si non, expliquez pourquoi.
\item Est-il possible de contrôler en optimisant le débit aval et en évitant les blocages et les situations critiques ?
\item Si oui, donnez quelques caractéristiques de ce contrôleur, si non, expliquez pourquoi.
\end{enumerate}


\chapter{Le rapport}
Le rapport est sur 20 points.

\section{Rôle du fichier {\tt GNUmakefile} (2 points)}
Le makefile effectue les calcules suivants : \\
Pour tout les controleurs ( {\tt Ctrl}, {\tt CtrlVV}) et  pour toutes les pannes (0,1,2,3), il concaténe tout les fichiers \textit{alt} dans un seul et même fichier \textit{tank.alt}.
Puis il inititialise les valeurs \textit{NBPannes} et \textit{NomControleur} dans ces fichiers et met le résultat dans un fichier \textit{test.alt} .
De la même façon, un fichier  \textit{system.spe} est créé et copié dans  \textit{test.spe} avec plusieurs variables initialisées. \\
Ensuite pour toutes les iterations (0,1,2,3,4,5),le controleur est modifié en fonction de l'itération et le calcul est lancé.
Les fichiers obtenus sont comparés et le programme coupe l'itération lorsque l'évolution stagne, c'est à dire quand deux itérations produisent un même résultat.

\section{Rôle de la constante {\tt nbFailures} et de l'assertion associée (1 point)}
La constante \textit{nbFailures} est une constante correspondant au nombre de vannes pouvant tomber en panne dans le système. Une vanne en panne est une vanne qui reste pour toujours dans son état actuel. Le système en possède trois et donc \textit{nbFailures} peut prendre les valeurs 0, 1, 2 ou 3 selon le nombre de défaillance étudiées dans le système. La contrainte $nbFailures >= (V [0].fail + V [1].fail + V [2].fail)$ présente dans le système permet de s’assurer que le nombre de défaillances dans le système actuel ne dépasse pas le nombre de défaillances autorisées. Par exemple, si on test le système avec une seule défaillance et qu'une vanne est déjà défaillante (V[x].fail = 1), cette contrainte va empécher les autres vannes de disfonctionner. 

\section{Résultats avec le contrôleur initial {\tt Ctrl}}
\subsection{Calcul d'un contrôleur}
\subsubsection{Avec 0 défaillance (0.5 point)}
\lstinputlisting{Res/System0FCtrl.res}
\lstinputlisting{Res/System0FCtrl0F1I.res}
\lstinputlisting{Res/System0FCtrl0F2I.res}
%\lstinputlisting{Res/System0FCtrl0F3I.res}
%\lstinputlisting{Res/System0FCtrl0F4I.res}
\paragraph{Interprétation des résultats}
Au départ on peut voir qu'un grand nombre de transition mènent à une fermeture de la valve aval (dec10 = 9500) et celle-ci est fermée dans 80 états (out0 = 80) 
soit dans environ un tier d'entre eux ce qui est beaucoup car on cherche à limiter la fermeture de la valve aval et le système ne prend en compte aucune défaillance. 
De plus, on peut voir qu'il y a 86 niveaux critiques et 86 situations redoutées, et environ un tier des \textit{targets} mènent à un coup gagnant.
Après une itération, le controleur est plus performant. On peut voir que la valve se retrouve fermée moins souvent (out0 = 26). De plus, il n'y a aucun blocage 
ni aucune situation redoutée ou critique. Le taux de coups gagnants est également bien meilleur.
Apres une seconde itération, le résultat obtenu est identique au résultat précédent et l'algorithme d'arrête. Il en sera de même pour l'ensemble des résultat suivant.


\subsubsection{Avec 1 défaillance (0.5 point)}
\lstinputlisting{Res/System1FCtrl.res}
\lstinputlisting{Res/System1FCtrl1F1I.res}
\lstinputlisting{Res/System1FCtrl1F2I.res}
\lstinputlisting{Res/System1FCtrl1F3I.res}
%\lstinputlisting{Res/System1FCtrl1F4I.res}
\paragraph{Interprétation des résultats}
% A COMPLETER
Suite à l'ajout d'une défaillance, on remarque tout d'abord que le nombre d'états et de transitions augmentent grandement ainsi que le nombre 
de niveaux critiques (NC = 329) et situations redoutées (SR = 329). Malgré tout, la proportion d'état où la valve aval est fermée est approximativement 
la même que lorsqu'il n'y a pas de défaillance (out0 = 300 ce qui correspond environ un tier des états). En revanche, le nombre de coup gagants par rapport 
au nombre de targets est faible (CCoupGagnant = 4950 pour any_t = 19450 soit environ un quart seulement) et un grand nombre d'états mènent à une situation où 
la valve aval est fermée.



\subsubsection{Avec 2 défaillances (0.5 point)}
\lstinputlisting{Res/System2FCtrl.res}
\lstinputlisting{Res/System2FCtrl2F1I.res}
\lstinputlisting{Res/System2FCtrl2F2I.res}
\lstinputlisting{Res/System2FCtrl2F3I.res}
\lstinputlisting{Res/System2FCtrl2F4I.res}
\paragraph{Interprétation des résultats}
% A COMPLETER

\subsubsection{Avec 3 défaillances (0.5 point)}
\lstinputlisting{Res/System3FCtrl.res}
\lstinputlisting{Res/System3FCtrl3F1I.res}
\lstinputlisting{Res/System3FCtrl3F2I.res}
\lstinputlisting{Res/System3FCtrl3F3I.res}
%\lstinputlisting{Res/System3FCtrl3F4I.res}
\paragraph{Interprétation des résultats}
% A COMPLETER

\subsection{Calcul des contrôleurs optimisés (2 points)}
\lstinputlisting{ControleursOpt/Optimisation.alt}
% A COMPLETER en expliquant en quoi consiste l'optimisation mise en place.
Dans la description du contrôleur en 1.1.3 il est écrit que le débit de la vanne aval doit être le plus important possible.
Le fichier optimisation.alt permet de définir des priorités sur le choix des actions. On va préférer incrémenter le débit aval 
plupart que le laisser au même niveau et on va préférer le laisser au même niveau plutôt que le décrémenter. De ce fait, on aura 
toujours le débit aval qui sera le plus élevé possible.
\paragraph{Avec 0 défaillance}
\lstinputlisting{Res/System0FCtrl0F2I_Opt.res}
% A COMPLETER en analysant les contrôleurs optimisés obtenus.

\section{Construction d'un contrôleur initial plus performant}
\subsection{Rôle du composant {\tt ValveVirtual}(2 points)}
% A COMPLETER en expliquant sa sémantique et son rôle.
Le composant {\tt ValveVirtual} correspond à une valve parfaite. Il est possible d'incrémenter ou de décrémenter l'ouverture 
de la valve afin de laisser passer plus ou moins d'eau. La valve se stoppe (reste dans son état) dans le cas où il y aurait une défaillance.

Contrairement au composant {\tt Valve}, il ne gère pas lui même s'il y a une défaillance. Il compare les valeurs \textit{rate} et 
\textit{rateReal}, cette dernière valeur étant donnée par {\tt CtrlrVV}. Dans le cas où ces deux dernières valeurs sont différentes,
 la valve reste inactive, mais n'a pas de variable propre au caractère de blocage comme dans \tt{Valve}.

\subsection{Rôle du composant {\tt CtrlVV} (5 points)}
% A COMPLETER en expliquant les mécanismes mis en oeuvre, leurs rôles et les avantages de ce contrôleur par rapport au précédent CtrlVV.

\section{Résultats avec le contrôleur {\tt CtrlVV}}
\subsection{Calcul d'un contrôleur}
\subsubsection{Avec 0 défaillance (0.5 point)}
\lstinputlisting{Res/System0FCtrlVV.res}
\lstinputlisting{Res/System0FCtrlVV0F1I.res}
\lstinputlisting{Res/System0FCtrlVV0F2I.res}
%\lstinputlisting{Res/System0FCtrlVV0F3I.res}
%\lstinputlisting{Res/System0FCtrlVV0F4I.res}
\paragraph{Interprétation des résultats}
Avec ce nouveau controleur munie de valves "virtuelles", et avec aucune défaillance, on obtient les même 
propriétés d'état qu'avec le controleur {\tt Ctrl}. Il n'y a que les propriétés de transitions qui changent, 
globalement elles sont toutes divisées par 2. On obtient cependant un ratio de coups gagnant / transitions possibles légèrement plus élevée.  \\
Lors de la seconde occurence et tout comme le le premier controleur, on obtient plus aucun Deadlock, Situation Redoutée ou Niveau Critique. 
On obtient cependant, moins de transitions, moins de transitions permettant de descrementer mais également moins de coup gagnants.

\subsubsection{Avec 1 défaillance (0.5 point)}
\lstinputlisting{Res/System1FCtrlVV.res}
\lstinputlisting{Res/System1FCtrlVV1F1I.res}
\lstinputlisting{Res/System1FCtrlVV1F2I.res}
\lstinputlisting{Res/System1FCtrlVV1F3I.res}
\lstinputlisting{Res/System1FCtrlVV1F4I.res}
\paragraph{Interprétation des résultats}
Lorsqu'il y a une défaillance, on obtient aucun deadlock, mais plusieurs Situations Redoutées et Niveaux Critiques. 
On a également moins de sommets contenant une vanne aval fermée, comparés aux autres niveaux d'ouverture. En ce qui concerne les 
transitions, on obtient en obtient nettement moins qu'avec le controleur {\tt Ctrl}. \\
Au fur et à mesure des itérations, le nombre de deadlocks et dr SR/NC diminue, jusqu'à être nul. On obtient également un nombre d'état où la valve aval est fermée
inférieur aux autres. Comme la tendance précedente, on obtient moins de transitions que dans le controleur précédent.

\subsubsection{Avec 2 défaillances (0.5 point)}
\lstinputlisting{Res/System2FCtrlVV.res}
\lstinputlisting{Res/System2FCtrlVV2F1I.res}
\lstinputlisting{Res/System2FCtrlVV2F2I.res}
\lstinputlisting{Res/System2FCtrlVV2F3I.res}
%\lstinputlisting{Res/System2FCtrlVV2F4I.res}
\paragraph{Interprétation des résultats}
% A COMPLETER

\subsubsection{Avec 3 défaillances (0.5 point)}
\lstinputlisting{Res/System3FCtrlVV.res}
\lstinputlisting{Res/System3FCtrlVV3F1I.res}
\lstinputlisting{Res/System3FCtrlVV3F2I.res}
\lstinputlisting{Res/System3FCtrlVV3F3I.res}
%\lstinputlisting{Res/System3FCtrlVV3F4I.res}
\paragraph{Interprétation des résultats}
% A COMPLETER

\subsection{Calcul des contrôleurs optimisés (2 points)}
% A COMPLETER en analysant les contrôleurs optimisés obtenus.
\paragraph{Avec 0 défaillance} \\
\lstinputlisting{Res/System0FCtrlVV0F2I_Opt.res}

\paragraph{Avec 1 défaillance} \\
\lstinputlisting{Res/System1FCtrlVV1F4I_Opt.res}

\paragraph{Avec 2 défaillances} \\
\lstinputlisting{Res/System2FCtrlVV2F3I_Opt.res}

\paragraph{Avec 3 défaillances} \\
\lstinputlisting{Res/System3FCtrlVV3F3I_Opt.res}

\section{Conclusion (2 points)}
% A COMPLETER

\end{document}
