\documentclass[a4paper]{book}
\usepackage{fullpage}

\usepackage[utf8]{inputenc}
\usepackage[T1]{fontenc}
\usepackage[francais]{babel}

\usepackage{latexsym}
\usepackage{fancyhdr}
\usepackage{makeidx}
\usepackage{graphics}
\usepackage{graphicx}
\usepackage{longtable}
\usepackage{moreverb}
\usepackage{listings}

\newcommand{\altarica}{{\sc AltaRica}}

\begin{document}

\title{Master 1, Conceptions Formelles\\
Projet du module \altarica\\
Synthèse (assistée) d'un contrôleur du niveau d'une cuve}

\date{}

\author{Jolliet Louis \and Sicardon Louis \and Vigneau Paul}

\maketitle

\chapter{Le sujet}
\input{tank}

\chapter{Le rapport}
Le rapport est sur 20 points.

\section{Rôle du fichier {\tt GNUmakefile} (2 points)}
% A COMPLETER en expliquant les enchainements des calculs effectués.

\section{Rôle de la constante {\tt nbFailures} et de l'assertion associée (1 point)}
La constante \textit{nbFailures} est une constante correspondant au nombre de vannes pouvant tomber en panne dans le système. Une vanne en panne est une vanne qui reste pour toujours dans son état actuel. Le système en possède trois et donc \textit{nbFailures} peut prendre les valeurs 0, 1, 2 ou 3. La contrainte $nbFailures >= (V [0].fail + V [1].fail + V [2].fail)$ présente dans le système permet de s’assurer que la valeur de \textit{nbFailures} se situe entre 0 et 3 et qu’elle correspond au nombre de vannes en panne.

\section{Résultats avec le contrôleur initial {\tt Ctrl}}
\subsection{Calcul d'un contrôleur}
\subsubsection{Avec 0 défaillance (0.5 point)}
\lstinputlisting{Res/System0FCtrl.res}
\lstinputlisting{Res/System0FCtrl0F1I.res}
\lstinputlisting{Res/System0FCtrl0F2I.res}
%\lstinputlisting{Res/System0FCtrl0F3I.res}
%\lstinputlisting{Res/System0FCtrl0F4I.res}
\paragraph{Interprétation des résultats}
% A COMPLETER

\subsubsection{Avec 1 défaillance (0.5 point)}
\lstinputlisting{Res/System1FCtrl.res}
\lstinputlisting{Res/System1FCtrl1F1I.res}
\lstinputlisting{Res/System1FCtrl1F2I.res}
\lstinputlisting{Res/System1FCtrl1F3I.res}
%\lstinputlisting{Res/System1FCtrl1F4I.res}
\paragraph{Interprétation des résultats}
% A COMPLETER

\subsubsection{Avec 2 défaillances (0.5 point)}
\lstinputlisting{Res/System2FCtrl.res}
\lstinputlisting{Res/System2FCtrl2F1I.res}
\lstinputlisting{Res/System2FCtrl2F2I.res}
\lstinputlisting{Res/System2FCtrl2F3I.res}
\lstinputlisting{Res/System2FCtrl2F4I.res}
\paragraph{Interprétation des résultats}
% A COMPLETER

\subsubsection{Avec 3 défaillances (0.5 point)}
\lstinputlisting{Res/System3FCtrl.res}
\lstinputlisting{Res/System3FCtrl3F1I.res}
\lstinputlisting{Res/System3FCtrl3F2I.res}
\lstinputlisting{Res/System3FCtrl3F3I.res}
%\lstinputlisting{Res/System3FCtrl3F4I.res}
\paragraph{Interprétation des résultats}
% A COMPLETER

\subsection{Calcul des contrôleurs optimisés (2 points)}
\lstinputlisting{ControleursOpt/Optimisation.alt}
% A COMPLETER en expliquant en quoi consiste l'optimisation mise en place.
Dans la description du contrôleur en 1.1.3 il est écrit que le débit de la vanne aval doit être le plus important possible.
Le fichier optimisation.alt permet de définir des priorités sur le choix des actions. On va préférer incrémenter le débit aval plupart que le laisser au même niveau et on va préférer le laisser au même niveau plutôt que le décrémenter. De ce fait, on aura toujours le débit aval qui sera le plus élevé possible.
\paragraph{Avec 0 défaillance}
\lstinputlisting{Res/System0FCtrl0F2I_Opt.res}
% A COMPLETER en analysant les contrôleurs optimisés obtenus.

\section{Construction d'un contrôleur initial plus performant}
\subsection{Rôle du composant {\tt ValveVirtual}(2 points)}
% A COMPLETER en expliquant sa sémantique et son rôle.

\subsection{Rôle du composant {\tt CtrlVV} (5 points)}
% A COMPLETER en expliquant les mécanismes mis en oeuvre, leurs rôles et les avantages de ce contrôleur par rapport au précédent CtrlVV.

\section{Résultats avec le contrôleur {\tt CtrlVV}}
\subsection{Calcul d'un contrôleur}
\subsubsection{Avec 0 défaillance (0.5 point)}
\lstinputlisting{Res/System0FCtrlVV.res}
\lstinputlisting{Res/System0FCtrlVV0F1I.res}
\lstinputlisting{Res/System0FCtrlVV0F2I.res}
%\lstinputlisting{Res/System0FCtrlVV0F3I.res}
%\lstinputlisting{Res/System0FCtrlVV0F4I.res}
\paragraph{Interprétation des résultats}
% A COMPLETER

\subsubsection{Avec 1 défaillance (0.5 point)}
\lstinputlisting{Res/System1FCtrlVV.res}
\lstinputlisting{Res/System1FCtrlVV1F1I.res}
\lstinputlisting{Res/System1FCtrlVV1F2I.res}
\lstinputlisting{Res/System1FCtrlVV1F3I.res}
\lstinputlisting{Res/System1FCtrlVV1F4I.res}
\paragraph{Interprétation des résultats}
% A COMPLETER

\subsubsection{Avec 2 défaillances (0.5 point)}
\lstinputlisting{Res/System2FCtrlVV.res}
\lstinputlisting{Res/System2FCtrlVV2F1I.res}
\lstinputlisting{Res/System2FCtrlVV2F2I.res}
\lstinputlisting{Res/System2FCtrlVV2F3I.res}
%\lstinputlisting{Res/System2FCtrlVV2F4I.res}
\paragraph{Interprétation des résultats}
% A COMPLETER

\subsubsection{Avec 3 défaillances (0.5 point)}
\lstinputlisting{Res/System3FCtrlVV.res}
\lstinputlisting{Res/System3FCtrlVV3F1I.res}
\lstinputlisting{Res/System3FCtrlVV3F2I.res}
\lstinputlisting{Res/System3FCtrlVV3F3I.res}
%\lstinputlisting{Res/System3FCtrlVV3F4I.res}
\paragraph{Interprétation des résultats}
% A COMPLETER

\subsection{Calcul des contrôleurs optimisés (2 points)}
% A COMPLETER en analysant les contrôleurs optimisés obtenus.
\paragraph{Avec 0 défaillance}\ \\
\lstinputlisting{Res/System0FCtrlVV0F2I_Opt.res}

\paragraph{Avec 1 défaillance}\ \\
\lstinputlisting{Res/System1FCtrlVV1F4I_Opt.res}

\paragraph{Avec 2 défaillances}\ \\
\lstinputlisting{Res/System2FCtrlVV2F3I_Opt.res}

\paragraph{Avec 3 défaillances}\ \\
\lstinputlisting{Res/System3FCtrlVV3F3I_Opt.res}

\section{Conclusion (2 points)}
% A COMPLETER

\end{document}
